\documentclass[a4paper,10pt]{book}
\usepackage[utf8x]{inputenc}

\begin{document}

\section{Methods}
To recover the SFR, age of stellar population, dust content, line of sight velocity distribution and metallicity from a galaxy, I used Monte Carlo optimization to efficiently do the necessary integrals for recovery of the posterior. I tried different Monte Carlo algorithms to sample the posterior. To calculate the posterior requires calculating integrals of the likelihood and priors. The posterior may not have an analytical solution This chapter will cover all the Monte Carlo methods used in this thesis to recover galactic parameters.

\subsection{Gibbs Sampling}

\subsection{Markov chain Monte Carlo}
Markov chain Monte Carlo (MCMC) is a class of sampling algorithms that use Markov chains to explore the posterior space. To use MCMC for parameter selection, the Metropolis-Hastings

\subsection{Reverse Jump Markov chain Monte Carlo}

Reverse Jump Markov chain Monte Carlo (RJMCMC) sampler is a way to allow the 
MCMC sampler to change the number of parameters that are being fitted and also 
can be used to fit to data while trying different models.

\subsection{Population-Based Reverse Jump Markov chain Monte Carlo}

\subsection{Population Monte Carlo}


\subsection{Nested Sampling}

\subsection{Hierarchical Models}


\end{document}
